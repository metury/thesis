\documentclass{article}

\usepackage{geometry}
\geometry{a4paper,total={170mm,257mm},left=20mm,top=20mm}

\usepackage{babel}
\usepackage[utf8]{inputenc}

\usepackage{amsmath}
\usepackage{amsthm}
\usepackage{amssymb}
\usepackage[dvipsnames]{xcolor}
\usepackage{fontawesome5}
\usepackage{hyperref}
\hypersetup{colorlinks=true, linkcolor=OrangeRed}

\theoremstyle{plain}
\newtheorem{thm}{Theorem}
\newtheorem{lemma}[thm]{Lemma}
\newtheorem{claim}[thm]{Claim}
\newtheorem{observ}[thm]{Observation}

\theoremstyle{plain}
\newtheorem{defn}{Definition}
\newtheorem{problem}{Problem}

\theoremstyle{remark}
\newtheorem*{cor}{Corollary}
\newtheorem*{rem}{Remark}
\newtheorem*{example}{Example}

\title{Notes on connected cuts}
\author{Tomáš Turek}
\date{\today}

\begin{document}
	\maketitle
	
	\tableofcontents
	
	\section{Preliminaries}
	
	We define graph $G = (V,E)$ as usual. Then we talk about edges defined by a cut in the following way. For vertices $S \subseteq V$ we define $E(S, V \setminus S) = \{e \in E \text{ s.t. } |e \cap S| = 1\}$, then the size of a cut is $e(S, V \setminus S) = |E(S, V \setminus S)|$. Also we will talk about the induced subgraphs which will be denoted as $G[S]$ for some vertices $S \subseteq V$. Just to recall that graph is called connected if in between every pair of vertices exist a walk. Also in most cases we will be considering graphs which are connected, but sometimes this is not necessary.
	
	
	\section{Connected cuts definitions}
	
	We will now proceed to some definitions of cuts which are in one way or another connected. We will start simply and build on that.
	
	\subsection{Single cuts}
	
	\begin{defn}[Connected cut]
		For a connected graph $G = (V,E)$ we define connected cut as $S \subseteq V$ for which $G[S]$ is connected. The cut itself is $E(S, V \setminus S)$. Later on we may exchange if we will talk about vertices or edges.
		\label{def-connected-cut}
	\end{defn}
	
	Now we will like to minimize the size of the cut, i.e. the value of $e(S, V \setminus S)$. Sometimes we may even define a connected cut with specific source vertex. This can be formulated by the next definition \ref{def-connected-s-cut}.
	
	\begin{defn}[Connected $s$ cut]
		For a connected graph $G = (V,E)$ and given vertex $s \in V$ we define connected cut as $S \subseteq V$ for which $G[S]$ is connected and also $s \in S$.
		\label{def-connected-s-cut}
	\end{defn}

	This is pretty much the same problem as in previous definition \ref{def-connected-cut}. Note that we would also like to minimize the size of the cut, i.e. $e(S, V \setminus S)$. And if we can solve it for the connected $s$ cut we may also apply it for all $s \in V$ to get the value for general connected cut.
	
	Some may already know that commonly used cut is for defined source and target distinct vertices. We could also use it in our case and only extend the previous definition by saying that $t \notin S$. It is somewhat tempting to also require that $G[V \setminus S]$ is supposed to be also connected. Therefore we can get the full connected $s-t$ cut.
	
	\begin{defn}[Connected $s-t$ cut]
		For a connected graph $G = (V,E)$ and given two distinct vertices $s, v \in V$ we define connected $s-t$ cut as $S \subseteq V$ for which all following properties hold.
		
		\begin{enumerate}
			\item $s \in S$ and $t \notin S$.
			\item Both $G[S]$ and $G[V \setminus S]$ are connected.
		\end{enumerate}
		\label{def-connected-s-t-cut}
	\end{defn}

	\subsection{Multi commodity cuts}

	Now we can furthermore generalize the notion of connected cuts to multi-way connected cuts.
	
	\begin{defn}[Multi-way connected cut]
		For a connected graph $G = (V,E)$ and pairwise distinct vertices $s_1, s_2, \dots, s_k \in V$ for $k \in \mathbb{N}$ we define connected cut as partition $\mathcal{V} = \{V_1, V_2, \dots, V_k\}$ of vertices (that is $\bigcup_{i = 1, \dots, k} V_i = V$ and for $i \neq j$ $V_i \cap V_j = \emptyset$) such that the following holds:
		
		\begin{enumerate}
			\item $\forall i \in [k]: s_i \in V_i$ and
			\item $\forall i \in [k]: G[V_i]$ is connected.
		\end{enumerate}
	\end{defn}
	
	In this specific definitions we may look at our problem from two perspective. Those two options can be seen by the optimization function for the given problem. Sum version is generally more easy to find the solution, or at least very good approximation. On the other hand optimizing over the max function can be way more tricky. Observe that the sum size is already computed with multi-commodity cut.
	
	\begin{itemize}
		\item \textit{Sum} size as $\sum_{i < j} E(V_i, V_j)$.
		\item \textit{Max} size as $\max_{i \in[k]} E(V_i, V \setminus V_i)$.
	\end{itemize}
	
	 Also we may define \textbf{Flexible multi-way connected cut} as relaxing the previous problem. That is the partition will have $l$ partitions where $0 < l \leq k$ and only $l$ sources are representing their partition. So $\forall i \in [l] , \exists k : s_k \in V_l$.
	
	\subsection{Other cuts}
	
	Now we can even further increase the number of requirements. In this case to the size of $|S|$. Now we will also state what is the optimization function and hence declare a problem.
	
	\begin{problem}[$k$-connected cut]
		For a connected graph $G = (V,E)$ we say $S \subseteq V$ is $k$ connected cut such that all properties hold:
		
		\begin{enumerate}
			\item $|S| = k$.
			\item $G[S]$ is connected.
			\item And we want to minimize $e(S, V \setminus S)$.
		\end{enumerate}
	\end{problem}

	From algorithmic perspective we may also have given source vertex $s \in V$, but as it was stated before we may solve the general case by running $|V|$ times the algorithm for the problem with source.
	
	Note that choosing only two properties from all three can be computed. If we skip the very first one, we may use the result from Garg, which states a linear program having all vertices as such result. Excluding the second one can be also computed via some approximation algorithm for bisection. And Overlooking the last one we just use some search, because we don't care about the size of the result.
	
	\section{Absorptive flow}
	
	We will be now talking about an absorptive flow. Firstly we will state the problem in a common sense. For a graph and a source we get a flow which flows through the graph and every time it goes through a vertex some of the flow gets absorbed into the given vertex. After that we will define a cut which is induced by such flow and later on state an integer program and its linear approximation. Now we properly state the instance.
	
	\begin{defn}[Absorptive flow]
		For a graph $G = (V,E)$ and a vertex $s \in V$, also called the \textit{source}, and for $k \in \mathbb{N}$, such that $|V| \geq k$, we define \textbf{absorptive flow} as a tuple of functions denoted as $(f_V, f_E)$, where $f_V : V \to \mathbb{R}$ and $f_E : E \to \mathbb{R}$. Now these two function must have these properties.
		
		\begin{enumerate}
			\item $\sum_{v \in V} f_V (v) = k$, that is every part of the flow gets absorbed,
			\item $f_V(s) = 1$,
			\item $\forall v \in V : 0 \leq f_V(v) \leq 1$, so all vertices have some limits,
			\item $\sum_{v \in V, (s,v) \in E} f_E(s,v) = k-1$, the flow starts from the source,
			\item $\forall e \in E : 0 \leq f_E(e)$, the flow has to be non-negative, but can be unlimited,
			\item $\forall v \in V \setminus \{s\}: \sum_{u \in V, (u,v) \in E} f_E((u,v)) = \sum_{u \in V, (v,u) \in E} f_E((v,u)) + f_V(v)$, thus the whole flow continue unless part of it is absorbed.
		\end{enumerate}
	\end{defn}
	
	One can already see that it resembles a linear program. Some can expect we would define a size of the flow, but in this special instance we won't be defining it, since the main purpose is to look at the cut, which is defined by the flow. So now we will define the cut.
	
	\subsection{Induced cut by the absorptive flow}
	
	Firstly we will define $S \subseteq V$ as the vertices which have nonzero function $f_V$, that is $\forall v \in S : f_V(s) > 0$. Then the \textbf{induced cut} defined by absorptive flow is defined as $E(S, V \setminus S)$ and its size as $e(S, V \setminus S)$. We will furthermore want to minimize the size of such cut.
	
	So far the only property is that $s \in S$, which can be seen only from the definition. Next observations come from the linear program and its properties.
	
	
	\section{Integer program}
	
	In this section we will establish the linear program which works with this flow and its cut.
	
	\subsection{Variables}
	
	Firstly we declare the variables for edges and for vertices.
	
	$$
	f_v = \left\{\begin{array}{l l}
		1 & \text{if it absorbs the flow} \\
		0 & \text{otherwise}
	\end{array}
	\right.
	$$
	
	$$
	f_{uv} \in [0,k] \text{ is for the amount of flow on the edge } uv.
	$$
	
	$$
	x_{uv} = \left\{\begin{array}{l l}
		1 & \text{if } uv \in E(S, V \setminus S)\\
		0 & \text{otherwise}
	\end{array}
	\right.
	$$
	
	See that these variables arise only from the definition of the problem.
	
	\subsection{Constraints}
	
	Now we need to state the constraints. Firstly set the connection between the absorbed vertices and the cut.
	
	$$
	\begin{array}{r l}
		x_{uv} \geq f_u - f_v & \forall \{uv\} \in E\\
		x_{uv} \geq f_v - f_u & \forall \{uv\} \in E
	\end{array}
	$$
	
	Which is basically that $x_{uv} \geq |f_u - f_v|$. Next we have to set the flow, so its properties hold.
	
	$$
	\begin{array}{r l}
		\sum_{v \in V, sv \in E} f_{sv} = k-1 \\
		f_s = 1 \\
		\sum_{u \in V, uv \in E} f_{uv} = \sum_{u \in V, vu \in E} f_{vu} + f_v & \forall v \in V, s \neq v \\
		\sum_{u \in V} f_u = k \\
		f_{v} \geq \frac{1}{\deg(v) \cdot(k-1)} \sum_{u \in V, \{uv\} \in E} f_{uv} & \forall v \in V \setminus \{s\}
	\end{array}
	$$
	
	\subsection{Optimization function}
	
	Lastly the optimization function will be to minimize the flow throughput and number of cut edges.
	
	$$
	\min \sum_{e \in E} x_e %+ \Delta f_e
	$$
	
	\subsection{The whole formulation}
	
	\begin{equation}
		\begin{array}{r l}
			\min \sum_{e \in E} x_e \\
			x_{uv} \geq f_u - f_v & \forall \{uv\} \in E\\
			x_{uv} \geq f_v - f_u & \forall \{uv\} \in E\\
			\sum_{v \in V, sv \in E} f_{sv} = k-1 \\
			f_s = 1 \\
			\sum_{u \in V, uv \in E} f_{uv} = \sum_{u \in V, vu \in E} f_{vu} + f_v & \forall v \in V, s \neq v \\
			\sum_{u \in V} f_u = k \\
			f_{v} \geq \frac{1}{\deg(v) \cdot(k-1)} \sum_{u \in V, \{uv\} \in E} f_{uv} & \forall v \in V \setminus \{s\} \\
			f_v \in \{0,1\} & \forall v \in V \\
			f_{uv} \in \mathbb{R}^+ & \forall \{u,v\} \in E \\
			x_{uv} \in \{0,1\} & \forall \{u,v\} \in E
		\end{array}
	\end{equation}
	
	\subsection{Properties}
	
	Lets talk about some crucial properties of this integer program.
	
	\begin{observ}
		Every vertex in the flow absorb.
	\end{observ}

	\begin{proof}
		See that due to the last constraint whenever a flow goes inside the vertex we must set the vertex to absorb some portion of the flow. Exactly 1 in the case of integer program.
	\end{proof}

	\begin{observ}
		$f_V$ defines a connected induced subgraph of $G$.
	\end{observ}

	\begin{proof}
		Since $f_s = 1$ we know that $s$ is inside the $G[S]$. For contradiction assume there is a vertex $v \in S$ which is not connected to the source $s$. Because $f_v = 1$ we must have that there is a flow over this vertex due to the fifth constraint. And since there is a flow to the vertex $v$ which starts in $s$ there is also a walk from $s$ to $v$, therefore no such vertex $v$ exists and all vertices are connected to $s$ and hence they induce connected subgraph.
	\end{proof}

	\begin{observ}
		We have a minimum connected $s$ cut from the optimal integer value of the lp formulation $x^\ast$.
	\end{observ}

	Now it is also a time to talk about integrality, moreover what will happen if we switch to lp relaxation.
\end{document}